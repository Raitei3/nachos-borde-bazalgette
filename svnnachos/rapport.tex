\documentclass[10pt]{article}

\usepackage[T1]{fontenc}
\usepackage[utf8]{inputenc}
\usepackage[francais]{babel}
\usepackage{hyperref}
\usepackage{url}
\usepackage{color}
\usepackage[usenames,dvipsnames]{xcolor}
\usepackage{verbatim}
\usepackage{tocbibind}
\usepackage{array}

\definecolor{vert2}{rgb}{0.09,0.30,0.12}
\definecolor{red2}{rgb}{0.43,0.05,0.1}
\definecolor{blue2}{rgb}{0.1,0.05,0.5}

\begin{document}
\title{Projet système : Nachos TP 2}
\author{Martin BAZALGETTE, Antoine BORDE}
\maketitle

\newpage
\tableofcontents
\newpage

\section{Bilan}
L'objectif de ce TP était de permettre l'exécution de threads simultanés.
Nous avons fini toutes les parties excepté la dernière concernant la mise en place de sémaphores pour l'utilisateur, car nous avons manqué de temps.
Nous avons également implémenté les appels système \textcolor{blue2}{PutInt} et \textcolor{blue2}{GetInt},
que nous n'avions pas compris lors du premier TP. Ainsi que corrigé toutes les erreurs qui
nous ont été signaler lors de la remise du rapport.
\newline
Tout ce que nous avons implémenté fonctionne, nous pouvons donc lancer autant de threads que nous voulons qui se termineront automatiquement.
D'autre part, bien que nous nous attendions à devoir revenir sur la partie précédente pour corriger des bugs inattendus, nous n'avons eu aucune mauvaise surprise et n'avons donc pas eu à modifier le code du TP1.

\subsection{Découverte du fonctionnement des threads Nachos}
Les 2 premières actions de la partie I nous ont permis de comprendre le fonctionnement des threads de Nachos. En effet nous avons repéré que les threads étaient initialisés dans le constructeur de la classe Thread. Nous avons également vu que la pile des threads Nachos se trouvait dans un espace mémoire réservé au noyau.
Par la suite, nous avons compris comment l'espace mémoire des threads était géré grâce aux objets AddrSpace, et comment ils se terminaient, en sautant à l'adresse de l'appel système \textcolor{blue2}{Exit}.

\subsection{Implémentation d'un thread utilisateur unique}
Une fois le fonctionnement des threads compris, nous sommes passés à l'implémentation d'un thread utilisateur.
Pour ce faire, nous avons dû ajouter un appel système \textcolor{blue2}{ThreadCreate}, qui créé un thread utilisable par l'utilisateur. Le thread lancera l'exécution d'une fonction définie par l'utilisateur.
\newline
Nous avons commencé par définir la fonction \textcolor{red2}{do\_ThreadCreate}, qui crée un thread Nachos, et récupère la fonction de l'utilisateur et ses arguments. On place ensuite le thread dans la file d'attente de l'ordonnanceur via la fonction \textcolor{red2}{Start}.
Nous avons ensuite défini une fonction \textcolor{red2}{StartUserThread}, qui est lancée par le thread Nachos, lorsque ce dernier est activé par l'ordonnanceur.
La fonction \textcolor{red2}{StartUserThread} initialise les registres et alloue un espace dans la pile au thread l'ayant appelée. Elle lance par la suite \textcolor{red2}{Machine->Run} pour lancer la console.
Une fois le thread utilisateur lancé, il nous fallait définir un appel système \textcolor{blue2}{ThreadExit} pour terminer le thread et libérer l'espace de la pile.
Cet appel invoque la méthode \textcolor{red2}{Finish} de la classe Thread.
\newpage
\subsection{Gestion de plusieurs threads}
Pour l'instant, si on essaye de lancer plusieurs threads simultanément, ils se chevauchent et s'écrasent, car l'espace qu'on leur fournit est exactement le même.
Pour résoudre ce problème, il nous était demandé dans un premier temps de verrouiller les traitements de synchconsole grâce à des sémaphores. Ainsi, le deuxième thread attend que le premier soit terminé pour se lancer. Il faut protéger tous les appels systèmes, notamment \textcolor{blue2}{SynchPutString} et \textcolor{blue2}{SynchGetString}, car même si elles font appel à \textcolor{blue2}{SynchPutChar} et \textcolor{blue2}{SynchGetChar} qui sont protégées, ces appels se font par plusieurs threads simultanément et tout est constamment écrasé.
\newline
Nous avons ensuite du améliorer l'appel système \textcolor{blue2}{ThreadExit}, car Nachos ne terminait pas correctement si le thread principal et le thread utilisateur appelaient tous les deux \textcolor{blue2}{ThreadExit}. Nous avons
résolu ce problème en plaçant \textcolor{red2}{Interrupt->Halt} dans la fonction \textcolor{red2}{do\_ThreadExit}.
\newline
Par la suite, il nous était demandé de permettre l'allocation de plusieurs threads dans des endroits différents de la pile, sans se soucier de la réutilisation de l'espace libéré par les threads terminés.
Cette méthode permet de lancer jusqu'a 3 threads utilisateur, car la pile n'est pas assez grande pour permettre à un quatrième thread de se lancer. Il y a donc 4 espaces dans la pile qui peuvent être utilisés par un thread.
Le nombre de threads que l'utilisateur peut lancer est donc limité
par la taille de la pile
\newline
Pour pallier cette limite , il nous était demandé dans la partie suivante de permettre la réutilisation de l'espace pile libéré.
Pour cela, nous avons utilisé la classe BitMap, qui gère la disponibilité des 4 espaces de la pile.
Grâce à cette technique, l'utilisateur peut désormais lancer autant
de threads qu'il le souhaite.

\subsection{Terminaison automatique}
La dernière chose que nous avons implémentée permet la terminaison automatique des threads. Jusque là, quand un thread n'appelait pas \textcolor{blue2}{ThreadExit}, Nachos ne terminait pas correctement.
Pour que les threads terminent automatiquement, nous avons rajouté un argument invisible pour l'utilisateur à \textcolor{red2}{do\_ThreadCreate}. Cet argument correspond à l'adresse de l'appel système \textcolor{blue2}{ThreadExit}.
On place ensuite cette adresse dans le registre 31, qui correspond à l'adresse de retour de l'appel système. Ainsi, dès qu'un thread termine sa tâche, il appelle \textcolor{blue2}{ThreadExit} automatiquement.

\newpage
\section {Points délicats}

Même si cette partie 2 était un peu plus compliquée, les difficultés étaient les mêmes
que pour la partie 1. Elles se situent principalement dans la compréhension du fonctionnement
de Nachos et dans la capacité à comprendre l'énoncé et à suivre le guide proposé sans
partir dans des implémentations compliquées et contre productives. Il faut aussi
bien prendre le temps de réfléchir avant d'implémenter, car tout ce qui a été fait dans
cette partie se répercutera sur la suivante (la gestion de plusieurs processus).
\newline
Bien évidemment c'est la gestion de plusieurs threads simultanés qui fut la plus délicate
à implémenter.

\subsection{Le multithreading}
\subsubsection{Sémaphores et Locks}
La première difficulté a été de comprendre le fonctionnement des sémaphores et des locks.
Dans un premier temps nous n'arrivions pas à savoir comment implémenter les locks, nous sommes
donc partis sur une implémentation basique à base de sémaphores. Cela nous a permis d'avancer
rapidement et nous avons naturellement mieux compris ce qu'était un sémaphore.
Nous avons rapidement constaté que toutes les protections que nous mettions en place ne
devaient laisser passer qu'un seul thread à la fois. Nous avons compris à ce moment-là qu'un
lock n'est rien d'autre qu'un sémaphore qui ne distribue qu'un seul jeton.
Nous avons donc remplacé les sémaphores par des locks dans notre implémentation.

\subsubsection{Les piles des threads}

Instancier et exécuter plusieurs threads ne constitue pas de difficulté en soit.
Faire en sorte qu'ils ne s'écrasent pas les uns les autres est un autre problème.
Nachos ne possède en effet que peu d'espace alloué à sa pile utilisateur.
On ne peut donc pas raisonnablement gérer plus de quatre threads simultanément.
Pour gérer l'allocation des piles des différents threads, nous avons utilisé la classe
bitmap fournie par Nachos. Nous avons également doté la classe thread d'une nouvelle donnée
qui est sa position sur la bitmap.
Cela nous a semblé être la solution la plus simple et
la plus efficace à condition de prendre garde à respecter le principe d'encapsulation.
\newline
Dans un premier temps, lorsque le thread principal passe pour la première fois dans \textcolor{red2}{do\_ThreadCreate}
nous regardons si la bitmap a été instanciée. Si ce n'est pas le cas, nous l'instancions
et marquons le thread principal à la première place.
Après cela, chaque thread instancié va prendre le premier slot libre disponible dans
la bitmap et mémoriser le numéro de ce slot. Puis il appelle \textcolor{red2}{allocateUserStack} avec ce
numéro pour récupérer l'espace d'adressage de la pile.
Il libère ensuite ce numéro dans \textcolor{blue2}{ThreadExit}.

\subsubsection{Protection des threads}

À partir de là, il reste encore un problème majeur à gérer. En effet nous n'avons absolument
pas la main sur l'ordonnanceur et nous ne pouvons pas prédire le moment où les threads
vont essayer de se lancer. Il faut donc mettre en attente les threads qui voudraient
s'allouer une pile avant que des places soit libérée. Au début nous avions décidé
d'appeler \textcolor{red2}{Thread->yield} qui remet le thread dans la liste d'attente de l'ordonnanceur.
\newline
Mais nous avons appris que cette technique impliquait de réveiller régulièrement les
threads en attente pour rien, ce qui gâchait du temps d'exécution.
Nous avons donc créé un gros sémaphore qui laisse passer 3 threads (le nombre maximum de threads
simultanés -1 pour le main) et dont l'entrée se situe au début de \textcolor{red2}{StartUserThread}, et la sortie dans \textcolor{red2}{do\_ThreadExit}.

\subsection{Terminaison automatique}
Un autre gros souci nous est arrivé lorsque nous avons voulu implémenter la terminaison
automatique des threads. Le thread principal finit par défaut par l'appel système \textcolor{blue2}{Exit}
qui interrompt la machine.
Nous avons envisagé deux solutions :
La première consistait à utiliser une variable pour compter le nombre de threads utilisateur actifs
et d'appeler \textcolor{red2}{Thread->yield} tant que celle-ci n'est pas à 0 pour faire attendre le thread principal
au niveau de l'appel système \textcolor{blue2}{Exit}. Nous avons décidé d'écarter cette solution, car yield()
consomme du temps (même si ce n'est que sur le thread principal)
La deuxième était d'appeler \textcolor{blue2}{ThreadExit} dans \textcolor{blue2}{Exit} et d'utiliser la bitmap pour appeler interrupt-Halt()
lorsque celle-ci est vide.


\section{Limitations}

Cette partie du rapport est encore une fois la plus difficile à rédiger. En effet
même si notre compréhension globale du fonctionnement de Nachos a bien évolué dans
cette partie, il reste très difficile de savoir où se situent les défauts de notre
implémentation (dans la mesure ou nos tests, fonctionnent). D'autant qu'il s'agit surtout
de savoir si nos choix d'implémentation sont les bons pour la suite du projet (la gestion
de plusieurs processus) et que cela reste très difficile à certifier.
\newline
Dans l'ensemble nous dirions que notre implémentation se veut simple et fonctionnelle.
Nous pensons être allés à l'essentiel sans nous compliquer la vie et avoir réussi à faire
un code qui fonctionne en collant au plus près des conseils du tp.
Nous n'avons pas repéré de bug, les tests effectués terminent tous convenablement. Nous gérons plusieurs
cas de figure notamment ceux où le thread principal appelle un thread qui appelle un thread.
\newpage
Nous avons néanmoins une petite fuite mémoire que nous n'arrivons pas a
supprimer (un block possiblement perdu d'âpres valgrind).
Cependant notre inquiétude majeure est de savoir si nous avons fait les bons choix d'implémentation
pour la suite du TP. En effet nous avons énormément centralisé le fonctionnement des
threads et de tout ce qui les entoure dans les fonctions \textcolor{red2}{do\_ThreadExit}
\textcolor{red2}{do\_ThreadCreate} et \textcolor{red2}{StartUserThread} du fichier \textcolor{vert2}{userthread.cc}.
De l'instanciation et l'utilisation des locks et des sémaphores à la bitmap en passant par
l'interruption même de la machine lorsqu'il ne reste plus aucun thread à exécuter. Tout
se gère dans \textcolor{vert2}{userthread.cc}.
\newline
Nous relativisons malgré tout la gravité d'une erreur à ce niveau-là, car nous pensons que si
des modifications doivent être faites plus tard il s'agira plus de déplacer l'implémentation existante que de
la modifier, ce qui ne représente pas la même quantité de travail.

\section {Tests}

Pour vérifier le fonctionnement de notre implémentation, nous avons écrit 5 programmes de test.

Le premier, \textcolor{vert2}{thread.c}, nous a servi à tester notre tout premier thread utilisateur. Il se contente de vérifier que le thread se lance, et exécute la fonction. Nous avons testé plusieurs fonctions qui faisaient des appels système différents.
\newline
Le deuxième test, \textcolor{vert2}{bothExit.c} nous a permis de vérifier que le thread principal et le thread utilisateur puissent tous les deux appeler \textcolor{blue2}{ThreadExit} sans faire planter la console.
\newline
Nous avons implémenté un troisième test \textcolor{vert2}{mutli-thread-test.c} qui crée 30 threads qui exécutent la même fonction. Là encore, nous avons testé avec des fonctions faisant des appels système différents.
\newline
Notre quatrième test, \textcolor{vert2}{autoExit.c}, nous a permis de vérifier que l'appel à \textcolor{blue2}{ThreadExit} se faisait bien automatiquement. Le test lance donc plusieurs threads sans appeler directement \textcolor{blue2}{ThreadExit}.
\newline
Dans le cinquième et dernier test \textcolor{vert2}{threadCascade.c}, nous créons un thread utilisateur qui lui-même crée un thread utilisateur.




\end{document}
\grid
