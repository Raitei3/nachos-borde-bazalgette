\documentclass[12pt]{article}

\usepackage[T1]{fontenc}
\usepackage[utf8]{inputenc}
\usepackage[francais]{babel}
\usepackage{hyperref}
\usepackage{url}
\usepackage{color}
\usepackage[usenames,dvipsnames]{xcolor}
\usepackage{verbatim}
\usepackage{tocbibind}
\usepackage{array}

\begin{document}
\title{Projet système : Nachos TP 1}
\author{Martin BAZALGETTE, Antoine BORDE}

\maketitle
\newpage
\tableofcontents
\newpage

\section{Bilan}
L'objectif de ce TP était d'ajouter quelques appels systèmes à NACHOS.
\newline
Pour pouvoir travailler en binôme efficacement, nous avons mis en place un dépot git possédant plusieurs branches, de manière à ce que chacun possède une version du projet personnelle, qu'il pourra modifier à son bon vouloir.
\newline
Nous avons commencé par découvrir le projet avec la partie II. Nous devions modifier légèrement le code de manière à comprendre comment Nachos gère ses entrées et ses sorties.
Nous avons appris l'existence de sémaphores, dont le rôle est primordial pour le bon fonctionnement de la console.
L'un des sémaphores intervient lorque l'on veut écrire sur la console, 
Ce fonctionnement étant asynchrone et assez primitif, nous l'avons amélioré dans la partie III, dont le but était d'implémenter une couche d'entrées/sorties synchrones.

\section {Points Délicats}


\section {Limitations}


\section {Tests}






\end{document}
\grid
