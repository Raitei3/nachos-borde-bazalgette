\documentclass[12pt]{article}

\usepackage[T1]{fontenc}
\usepackage[utf8]{inputenc}
\usepackage[francais]{babel}
\usepackage{hyperref}
\usepackage{url}
\usepackage{color}
\usepackage[usenames,dvipsnames]{xcolor}
\usepackage{verbatim}
\usepackage{tocbibind}
\usepackage{array}

\definecolor{vert2}{rgb}{0.09,0.30,0.12}
\definecolor{red2}{rgb}{0.43,0.05,0.1}
\definecolor{blue2}{rgb}{0.1,0.05,0.5}

\begin{document}
\title{Projet système : Nachos TP 2}
\author{Martin BAZALGETTE, Antoine BORDE}
\maketitle

\newpage
\tableofcontents
\newpage

\section{Bilan}
L'objectif de ce TP était de permettre l'exécution de threads simultanés.
Nous avons fini toutes les parties excepté la dernière concernant la mise en place de sémaphores pour l'utilisateur, car nous avons manqué de temps.
Nous avons également implémenté les appels système PutInt et GetInt,
que nous n'avions pas compris lors du premier TP. Ainsi que corrigé toutes les erreurs qui
nous ont été signaler lors de la remise du rapport.
Tout ce que nous avons implémenté fonctionne, nous pouvons donc lancer autant de threads que nous voulons qui se termineront automatiquement.
D'autre part, bien que nous nous attendions à devoir revenir sur la partie précédente pour corriger des bugs innatendus, nous n'avons eu aucune mauvaise surprise et n'avons donc pas eu à modifier le code du TP1.

\subsection{Découverte du fonctionnement des threads Nachos}
Les 2 premières actions de la partie I nous ont permis de comprendre le fonctionnement des threads de Nachos. En effet nous avons repéré que les threads étaient initialisés dans le constructeur de la classe Thread. Nous avons également vu que la pile des threads Nachos se trouvait dans un espace mémoire réservé au noyau.
Par la suite, nous avons compris comment l'espace mémoire des threads était géré grâce aux objets AddrSpace, et comment ils se terminaient, en sautant à l'adresse de l'appel système Exit.

\subsection{Implémentation d'un thread utilisateur unique}
Une fois le fonctionnement des threads compris, nous sommes passés à l'implémentation d'un thread utilisateur.
Pour ce faire, nous avons du ajouter un appel système ThreadCreate, qui créé un thread utilisable par l'utilisateur. Le thread lancera l'exécution d'une fonction définie par l'utilisateur.
Nous avons commencé par définir la fonction do\_ThreadCreate, qui créé un thread Nachos, et récupère la fonction de l'utilisateur et ses arguments. On place ensuite le thread dans la file d'attente de l'ordonnanceur via la fonction Start.
Nous avons ensuite défini une fonction StartUserThread, qui est lancée par le thread Nachos, lorsque ce dernier est activé par l'ordonnanceur.
La fonction StartUserThread initialise les registres et alloue un espace dans la pile au thread l'ayant appelée. elle lance par la suite machine->Run pour lancer la console.
Une fois le thread utilisateur lancé, il nous fallait définir un appel système ThreadExit pour terminer le thread et libérer l'espace de la pile.
Cet appel invoque la méthode finish de la classe Thread.

\subsection{Gestion de plusieurs threads}
Pour l'instant, si on essaye de lancer plusieurs threads simultanément, ils se chevauchent et s'ecrasent car l'espace qu'on leur fourni est exactement le même.
Pour résoudre ce problème, il nous était demandé dans un premier temps de vérouiller les traitements de synchconsole grâce à des sémaphores. Ainsi, le deuxième thread attend que le premier soit terminé pour se lancer. Il faut protéger toutes les fonctions, nottament SynchPutString et SynchGetString car même si elles font appel à SynchPutChar et SynchGetChar qui sont protégées, ces appels se font par plusieurs threads simultanément et tout est constamment écrasé.

Nous avons ensuite du améliorer l'appel système threadExit, car Nachos ne terminait pas correctement si le thread principal et le thread utilisateur appellaient tous les deux threadExit. Nous avons
résolu ce problème en plaçant interrupt->Halt() dans la fonction do\_ThreadExit.

Par la suite, il nous était demandé de permettre l'allocation de plusieurs threads dans des endroits différents de la pile, sans se soucier de la réutilisation de l'espace libéré par les threads terminés.
Cette méthode permet de lancer jusqu'a 3 threads utilisateur, car la pile n'est pas assez grande pour permettre à un quatrième thread de se lancer. Il y a donc 4 espaces dans la pile qui peuvent être utilisés par un thread.
Le nombre de threads que l'utilisateur peut lancer est donc limité
par la taille de la pile

Pour pallier cette limite , il nous était demandé dans la partie suivante de permettre la réutilisation de l'espace pile libéré.
Pour cela, nous avons utilisé la classe BitMap, qui gère la disponibilité des 4 espaces de la pile.
Grâce à cette technique, l'utilisateur peut désormais lancer autant
de threads qu'il le souhaite.

\subsection{Terminaison automatique}
La dernière chose que nous avons implémenté permet la terminaison automatique des threads. Jusque là, quand un thread n'appelait pas threadExit, Nachos ne terminait pas correctement.
Pour que les threads terminent automatiquement, nous avons rajouté un argument invisible pour l'utilisateur à do\_ThreadCreate. Cet argument correspond à l'adresse de l'appel système threadExit.
On place ensuite cette adresse dans le registre 31, qui correspond à l'adresse de retour de l'appel système. Ainsi, dès qu'un thread termine sa tâche, il appelle threadExit automatiquement.

\section {Tests}

Pour vérifier le fonctionnement de notre implémenatation, nous avons écrit 5 programmes de test.

Le premier, thread.c, nous a servi à tester notre tout premier thread utilisateur. Il se contente seulement de vérifier que le thread se lance, et exécute la fonction. Nous avons testé plusieurs fonctions qui faisaient des appels système différents.

Le deuxième test, bothExit.c nous a permis de vérifier que le thread principal et le thread utilisateur puissent tous les deux appeler ThreadExit sans faire planter la console.

Nous avons implémenté un troisième test mutli-thread-test qui créé 30 threads qui exécutent la même fonction. Là encore, nous avons testé avec des fonctions faisant des appels système différents.

Notre quatrième test, autoExit.c, nous a permi de vérifier que l'appel à ThreadExit se faisait bien automatiquement. Le test lance donc plusieurs threads sans appeler dirrectement ThreadExit.

Dans le cinquième et dernier test threadCascade.c, nous créons un thread utilisateur qui lui même crée un thread utilisateur.








\end{document}
\grid
